\chapter{Problema}
La ricerca di esperti è il compito di trovare esperti di un dato dominio.\\ La profilazione degli esperti è il compito di restituire una lista di argomenti sui quali una persona è ben informata e competente.\\
La ricerca di esperti fa uso di tool e di sw per trovare e valutare le competenze di un individuo.\\
La ricerca di esperti è una sfida per molte ragioni. Ad esempio, il volume di pubblicazioni di un esperto non è indice di esperienza.\\
Un altro problema è che la vera esperienza è rara e costosa, spesso l'accesso ai dati è controllato dall'esperto stesso.\\
C'è da tenere in considerazione che le competenze di un esperto cambiano continuamente e bisogna essere  consapevolevi di questa dinamica.\\
Nella ricerca degli esperti non bisogna mai fermarsi al primo esperto trovato, perché potrebbe non essere la persona più colta e preparata per quell'argomento.\\
Alcuni argomenti e topic generano più opinioni che conoscenza, quindi trovare il vero esperto può essere difficile.\\
Spesso c'è una mancanza di accesso alle informazioni sulle conoscenze passate dell'esperto.\\
Le difficoltà nell'analisi e nella ricerca degli esperti è che non ci sono degli standard che specificano i criteri e le qualifiche necessarie per particolari livelli di esperienza.\\
Spesso le soluzioni a problemi complessi spesso richiedono comunità di esperti e diverse gamme di competenze, per cui si devono riunire gli esperti per risolvere i problemi.\\
In sintesi, la ricerca di esperti è un compito complesso e difficile.

\section{Dataset}
I dataset utilizzati sono due: \textit{Yelp} e \textit{Stack Exchange}.

\begin{wrapfigure}{r}{0.4\textwidth}
    \centering
    \includegraphics[scale=0.1]{image/yelp.png}
    \label{fig:yelp}
\end{wrapfigure}
\subsection{Yelp}
Yelp è un social network in cui le persone si scambiano pareri, opinioni e “dritte” sui posti migliori del luogo in cui abitano e di quelli in cui vanno per lavoro, viaggio o altri motivi. Di conseguenza, uno dei punti di forza di Yelp è da un lato il crowdsourcing dei contenuti, ossia il fatto che il contenuto è creato dagli utenti, e dall’altro la potenza della community che genera ed interagisce sul sito.
Su Yelp non ci sono solo recensioni di alberghi e ristoranti: si possono recensire (e leggere recensioni) di qualunque attività, purchè sia accessibile al pubblico. Tutti i luoghi recensiti sono luoghi reali, che altre persone possono visitare a loro volta.\\
Il nostro dataset è costituito da tanti file JSON.
Il JSON (\textbf{JavaScript Object Notation}) è un formato per lo scambio di dati. Risulta facile da leggere e scrivere per le persone; mentre per le macchine è facile da generare e analizzare la sintassi.
JSON è un formato di testo completamente indipendente dal linguaggio di programmazione. È leggero per la memorizzazione e il trasporto dei dati. I dati sono in coppie nome/valore.\\
Ogni file è composto da un singolo tipo di oggetto, un oggetto JSON per linea. \newline
Nel dataset ci sono 5 file JSON e sono \textit{business, checkin ,review, tip, user}.\\
I file JSON da noi utilizzati sono \textit{business, review e user}.

\begin{itemize}
\item \textbf{Business:}
Contiene i dati di ogni locale, come ad esempio un ristorante, inclusi i dati sulla posizione, gli attributi e le categorie.
Ogni locale è identificato da un \textit{business\_id}.\\
Un campo importante è \textbf{categories}, che è composto da una serie di stringhe aziendali, esso classifica il topic. Con topic si intende l'argomento di una discussione.\\
Per esempio attraverso \textit{categories} possiamo conoscere se in un ristorante si mangiano pietanze messicane.\\
Un locale deve avere un \textit{nome} ed una posizione geografica identificata da \textit{address, city, state, postal\_code, latitude, longitude}.\\
Il campo \textit{review\_count} rappresenta il numero di recensioni che ha avuto il locale.\\
\textit{attributes} da informazioni aggiuntive sul locale, ad esempio se è presente un parcheggio privato che il cliente può utilizzare.\\
\textit{hours} indica i giorni e l' orario di apertura del locale.\\
\textit{stars} rappresenta il numero di stelle assegnato ad un locale.\\
\textit{is\_open } indica se un locale è ancora aperto.\\

\item \textbf{Review:}
Contiene le recensioni scritte dagli utenti.\\
Un campo di particolare interesse è \textbf{useful} che è composto da un numero intero e rappresenta il numero di voti che reputano utile la recensione. \'E stato utilizzato per selezionare le recensioni più rilevanti di un determinato topic.\\
\textit{useful} è considerato un campo con notevole peso nelle elaborazioni, perché è utile nella ricerca degli esperti.\\
Ogni recensione è identificata da \textit{review\_id}.
\textit{user\_id} identifica l'utente che ha scritto la recensione.
\textit{business\_id,} identifica il locale per cui è stata scritta la recensione.\\
Gli altri campi contenuti all' interno del file sono: \textit{stars, date, text, funny, cool}.\\

\item \textbf{User:}
Sono gli utenti che interagiscono con Yelp, essi scrivono le recensioni per un determinato locale (riguardante un certo topic).
Nel file sono salvati anche i metadati associati agli amici dell'utente.\\
Ogni utente è identificato da \textit{user\_id}, che è una stringa di caratteri.\\
Un campo distintivo è \textbf{elite} che indica gli anni in cui il recensore è stato un utente elite.
Gli elite vengono eletti una volta all'anno.
Gli utenti elite hanno avuto una particolare considerazione nelle elaborazioni per ricercare gli esperti.\\ 
Altri campi di interesse sono: \textbf{compliment\_list, compliment\_profile, fans, useful}.
Anche questi campi hanno contribuito alla ricerca di un utente compentente.\\
\textit{compliment\_list} è il numero di elenchi di complimenti che utente possiede.\\
\textit{compliment\_profile} rappresenta il numero utenti che si sono di complimentati con l'utente.\\
\textit{fans} è un intero ed è il numero dei fan che un recensore possiede.\\
\textit{useful} è un intero ed è il numero di voti dato ad un utente.\\
Gli altri campi che compongono il file user sono:
\textit{name, review\_count, yelping\_since, friends, funny, cool, average\_stars, compliment\_hot, compliment\_more, compliment\_profile, compliment\_cute, \\ compliment\_note, compliment\_plain, compliment\_cool,\\ compliment\_funny, compliment\_writer, compliment\_photos}.

\item \textbf{Checkin:}
Contiene i campi \textit{business\_id e date}, utilizzati rispettivamente per identificare
il locale e per salvare la data e l'ora del checkin.

\item \textbf{Tip:}
Sono i suggerimenti scritti da un utente su un'azienda. I suggerimenti sono più brevi e concisi delle recensioni.\\
I campi \textit{ business\_id, user\_id} identificano il locale a cui è rivolto il suggerimento e l'utente che lo ha scritto.\\
\textit{text e date} sono il contenuto del suggerimento e la data ed ora in cui è stato scritto.\\
\textit{compliment\_count} è numero di complimenti che ha avuto un suggerimento.
\item \textbf{Photo:}
Contiene le foto caricate dagli utenti riguardo ad un locale.
Il campo \textit{etichetta} indica cosa è stato scattato; per esempio l'utente potrebbe aver fotografato il cibo oppure aver fotografato l' ambiente del locale.
Gli altri campi sono \textit{photo\_id, business\_id, caption}.
\end{itemize}
Solo alcuni di questi file JSON sono stati utilizzati nel progetto, poiché solo alcuni sono di rilievanti per la ricerca degli esperti.
I file selezionati per le nostre elaborazioni sono: \textbf{Business}, \textbf{Review}, \textbf{User}.

\begin{wrapfigure}{r}{0.4\textwidth}
    \centering
    \includegraphics[scale=0.15]{image/stackexchange.png}
    \label{fig:stack}
\end{wrapfigure}
\subsection{Stack Exchange}
Stack Exchange è una rete di siti question-and-answer (Q\&A), in cui ciascuno di essi copre uno specifico topic di interesse e dove domande, risposte e utenti sono soggetti ad un processo di assegnazione di una certa reputazione.\\
Lo scopo primario di ogni sito di Stack Exchange è quello di permettere agli utenti di porre domande e rispondere ad esse.\\
Ogni sito tratta un determinato topic. Gli utenti possono votare sia le domande che le risposte e, attraverso questo processo, possono ricevere privilegi accumulando punti reputazione.\\\\
Il nostro dataset è costituito da tanti file zip quanti sono i siti (e di conseguenza i topic), in ognuno di essi (o nella maggior parte) ci sono 7 file xml: \textit{Badges, Comments, PostHistory, PostLinks, Posts, Users, Votes}.\\

\begin{itemize}
    \item \textbf{Badges}: sono delle etichette che possono essere assegnate ad un utente (ai suoi post, alle sue risposte o al suo profilo) e sono rappresentative delle sue attività sul sito. Per esempio \textit{Teacher} viene assegnato a qulle risposte che hanno uno score maggiore o uguale a 1, oppure \textit{Archaeologist}, che viene assegnato ad un utente che modifica un post inattivo da più di 6 mesi. I campi che contiene sono: \textit{Id, UserId, Name} (nome del badge) e \textit{Date} (data di assegnazione del badge).
    \item \textbf{Comments}: sono dei commenti ai post (domande o risposte). I campi che contiene sono: \textit{Id, PostId, Score, Text} e \textit{Creation Date}.
    \item \textbf{Posts}: sono le domande o le risposte scritte dagli utenti, distinti da un campo \textit{PostTypeId}, che è 1 quando si tratta di una domanda e 2 quando si tratta di una risposta. Gli altri campi sono: \textit{Id, ParentId} (presente solo se si tratta di una risposta) \textit{AcceptedAnswerId} (presente solo se si tratta di una domanda) \textit{CreationDate, Score} (dato dalla differenza tra il numero di persone che hanno trovato utile quel post e quelle che invece non lo hanno trovato soddisfacente), \textit{ViewCount} (numero di visualizzazioni), \textit{Body} (testo), \textit{OwnerUserId} (autore del post), \textit{Title, AnswerCount, CommentCount, FavouriteCount}.
    \item \textbf{PostHistory}: storia di un post, ovvero tutte le operazioni che un post ha subito come modifica del titolo o del corpo, chiusura, riapertura ecc. Ogni operazione è identificata da un id espresso dal campo \textit{PostHistoryTypeId}. Gli altri campi sono: \textit{PostId, CreationDate, UserId, Comment} e \textit{CloseReasonId} (id dell'eventuale motivo per cui un post è stato chiuso).
    \item \textbf{PostLinks}: collegamenti ad altri post o a duplicati dello stesso post. Questo è specificato dal campo \textit{PostLinkTypeId}. Altri campi sono: \textit{Id, CreationDate, PostId, RelatedPostId}.
    \item \textbf{Users}: sono gli utenti che interagiscono con i siti e quindi scrivono domande e risposte. Alcuni campi di questo file riguardano le generalità dell'utente come: \textit{DisplayName, EmailHash, WebsiteUrl, Location, Age} ed \textit{About me} (piccola descrizione che l'utente fa di se stesso).\\ 
    Il campo più importante e da tenere in considerazione è invece \textit{Reputation}, essa ci indica proprio la reputazione che l'utente si è guadagnato da quando si è iscritto a Stack Exchange, interagendo ai post (sia pubblicando domande che risposte). Questo valore viene calcolato sulla base di diversi punteggi che vengono assegnati all'utente a seconda della quantità di \textit{upvotes} e \textit{downvotes} che riceve ai suoi post. La reputazione acquisita dall'utente, inoltre, gli permette di avere diversi privilegi e sfruttare funzionalità all'interno del sito. Altri campi sono la \textit{CreationDate} e le \textit{Views} del profilo. 
    \item \textbf{Votes}: sono dei voti assegnati ad un post, che possono avere diversi significati a seconda del valore assunto dal campo \textit{VoteTypeId} (es. spam, offensive, favorite che caratterizzano il contenuto, oppure close, reopen, acceptedbyoriginator che ne denotano lo stato). Altri campi sono: \textit{PostId} e \textit{CreationDate}.
\end{itemize}
\vspace{1cm}
Come si evince dal contenuto di questi file xml, solo alcuni di essi abbiamo utilizzato, perché maggiormente significativi per il nostro obiettivo di trovare gli esperti riguardo un certo topic. I file selezionati per le nostre elaborazioni sono: \textbf{Posts} e \textbf{Users}.